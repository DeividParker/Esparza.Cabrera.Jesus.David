

\documentclass[12pt]{article}
\usepackage{amsmath}
\usepackage{latexsym}
\usepackage{amsfonts}
\usepackage[normalem]{ulem}
\usepackage{array}
\usepackage{amssymb}
\usepackage{graphicx}
\usepackage[backend=biber,
style=numeric,
sorting=none,
isbn=false,
doi=false,
url=false,
]{biblatex}\addbibresource{bibliography.bib}

\usepackage{subfig}
\usepackage{wrapfig}
\usepackage{wasysym}
\usepackage{enumitem}
\usepackage{adjustbox}
\usepackage{ragged2e}
\usepackage[svgnames,table]{xcolor}
\usepackage{tikz}
\usepackage{longtable}
\usepackage{changepage}
\usepackage{setspace}
\usepackage{hhline}
\usepackage{multicol}
\usepackage{tabto}
\usepackage{float}
\usepackage{multirow}
\usepackage{makecell}
\usepackage{fancyhdr}
\usepackage[toc,page]{appendix}
\usepackage[hidelinks]{hyperref}
\usetikzlibrary{shapes.symbols,shapes.geometric,shadows,arrows.meta}
\tikzset{>={Latex[width=1.5mm,length=2mm]}}
\usepackage{flowchart}\usepackage[paperheight=11.0in,paperwidth=8.5in,left=0.79in,right=0.79in,top=0.79in,bottom=0.79in,headheight=1in]{geometry}
\usepackage[utf8]{inputenc}
\usepackage[T1]{fontenc}
\TabPositions{0.49in,0.98in,1.47in,1.96in,2.45in,2.94in,3.43in,3.92in,4.41in,4.9in,5.39in,5.88in,6.37in,6.86in,}

\urlstyle{same}


 %%%%%%%%%%%%  Set Depths for Sections  %%%%%%%%%%%%%%

% 1) Section
% 1.1) SubSection
% 1.1.1) SubSubSection
% 1.1.1.1) Paragraph
% 1.1.1.1.1) Subparagraph


\setcounter{tocdepth}{5}
\setcounter{secnumdepth}{5}





\setlistdepth{9}
\renewlist{enumerate}{enumerate}{9}
		\setlist[enumerate,1]{label=\arabic*)}
		\setlist[enumerate,2]{label=\alph*)}
		\setlist[enumerate,3]{label=(\roman*)}
		\setlist[enumerate,4]{label=(\arabic*)}
		\setlist[enumerate,5]{label=(\Alph*)}
		\setlist[enumerate,6]{label=(\Roman*)}
		\setlist[enumerate,7]{label=\arabic*}
		\setlist[enumerate,8]{label=\alph*}
		\setlist[enumerate,9]{label=\roman*}

\renewlist{itemize}{itemize}{9}
		\setlist[itemize]{label=$\cdot$}
		\setlist[itemize,1]{label=\textbullet}
		\setlist[itemize,2]{label=$\circ$}
		\setlist[itemize,3]{label=$\ast$}
		\setlist[itemize,4]{label=$\dagger$}
		\setlist[itemize,5]{label=$\triangleright$}
		\setlist[itemize,6]{label=$\bigstar$}
		\setlist[itemize,7]{label=$\blacklozenge$}
		\setlist[itemize,8]{label=$\prime$}

\setlength{\topsep}{0pt}\setlength{\parindent}{0pt}
\renewcommand{\arraystretch}{1.3}


%%%%%%%%%%%%%%%%%%%% Document code starts here %%%%%%%%%%%%%%%%%%%%



\begin{document}
\par 
 \begin{tikzpicture}

\path (4.22in,-5.55in) node [shape=rectangle,draw,minimum height=10.64in,minimum width=8.27in,]{};

\end{tikzpicture}

\vspace{\baselineskip}
{\fontsize{28pt}{33.6pt}\selectfont \textbf{\textit{\textcolor[HTML]{00508F}{Calcular los parámetros de circuitos de activación de transistores de potencia}}}\par}\par


\vspace{\baselineskip}
{\fontsize{26pt}{31.2pt}\selectfont \textbf{Por: Jesús David Esparza Cabrera}\par}\par



%%%%%%%%%%%%%%%%%%%% Figure/Image No: 1 starts here %%%%%%%%%%%%%%%%%%%%

\begin{figure}[H]
	\begin{Center}
		\includegraphics[width=5.39in,height=5.87in]{./media/image1.png}
	\end{Center}
\end{figure}


%%%%%%%%%%%%%%%%%%%% Figure/Image No: 1 Ends here %%%%%%%%%%%%%%%%%%%%

\par


\vspace{\baselineskip}

\vspace{\baselineskip}

\vspace{\baselineskip}

\vspace{\baselineskip}

\vspace{\baselineskip}

\vspace{\baselineskip}

\vspace{\baselineskip}

\vspace{\baselineskip}

\vspace{\baselineskip}

\vspace{\baselineskip}

\vspace{\baselineskip}

\vspace{\baselineskip}

\vspace{\baselineskip}

\vspace{\baselineskip}

\vspace{\baselineskip}
{\fontsize{22pt}{26.4pt}\selectfont \textbf{Grado y grupo: 4-B}\par}\par

{\fontsize{22pt}{26.4pt}\selectfont \textbf{Materia: Sistemas electrónicos de interfaz}\par}\par


\vspace{\baselineskip}
{\fontsize{22pt}{26.4pt}\selectfont \textbf{EL TRANSISTOR DE POTENCIA}\par}\par


\vspace{\baselineskip}
El funcionamiento y utilización de los transistores de potencia es idéntico al de los transistores normales, teniendo como características especiales las altas tensiones e intensidades que tienen que soportar y, por tanto, las altas potencias a disipar. \par

Existen tres tipos de transistores de potencia: \par

\begin{itemize}
	\item bipolar. \par

	\item unipolar o FET (Transistor de Efecto de Campo). \par

	\item IGBT. 
\end{itemize}\par


\vspace{\baselineskip}


%%%%%%%%%%%%%%%%%%%% Table No: 1 starts here %%%%%%%%%%%%%%%%%%%%


\begin{table}[H]
 			\centering
\begin{tabular}{p{2.21in}p{1.22in}p{1.3in}}
\hline
%row no:1
\multicolumn{1}{|p{2.21in}}{\textbf{Parámetros}} & 
\multicolumn{1}{|p{1.22in}}{\textbf{MOS}} & 
\multicolumn{1}{|p{1.3in}|}{\textbf{Bipolar}} \\
\hhline{---}
%row no:2
\multicolumn{1}{|p{2.21in}}{Impedancia de entrada} & 
\multicolumn{1}{|p{1.22in}}{Alta (1010 ohmios)} & 
\multicolumn{1}{|p{1.3in}|}{Media (104 ohmios)} \\
\hhline{---}
%row no:3
\multicolumn{1}{|p{2.21in}}{Ganancia en corriente} & 
\multicolumn{1}{|p{1.22in}}{Alta (107)} & 
\multicolumn{1}{|p{1.3in}|}{Media (10-100)} \\
\hhline{---}
%row no:4
\multicolumn{1}{|p{2.21in}}{Resistencia ON (saturación)} & 
\multicolumn{1}{|p{1.22in}}{Media / alta} & 
\multicolumn{1}{|p{1.3in}|}{Baja} \\
\hhline{---}
%row no:5
\multicolumn{1}{|p{2.21in}}{Resistencia OFF (corte)} & 
\multicolumn{1}{|p{1.22in}}{Alta} & 
\multicolumn{1}{|p{1.3in}|}{Alta} \\
\hhline{---}
%row no:6
\multicolumn{1}{|p{2.21in}}{Voltaje aplicable} & 
\multicolumn{1}{|p{1.22in}}{Alto (1000 V)} & 
\multicolumn{1}{|p{1.3in}|}{Alto (1200 V)} \\
\hhline{---}
%row no:7
\multicolumn{1}{|p{2.21in}}{Máxima temperatura de operación} & 
\multicolumn{1}{|p{1.22in}}{Alta (200ºC)} & 
\multicolumn{1}{|p{1.3in}|}{Media (150ºC)} \\
\hhline{---}
%row no:8
\multicolumn{1}{|p{2.21in}}{Frecuencia de trabajo} & 
\multicolumn{1}{|p{1.22in}}{Alta (100-500 Khz)} & 
\multicolumn{1}{|p{1.3in}|}{Baja (10-80 Khz)} \\
\hhline{---}
%row no:9
\multicolumn{1}{|p{2.21in}}{Coste} & 
\multicolumn{1}{|p{1.22in}}{Alto} & 
\multicolumn{1}{|p{1.3in}|}{Medio} \\
\hhline{---}

\end{tabular}
 \end{table}


%%%%%%%%%%%%%%%%%%%% Table No: 1 ends here %%%%%%%%%%%%%%%%%%%%


\vspace{\baselineskip}
El IGBT ofrece a los usuarios las ventajas de entrada MOS, más la capacidad de carga en corriente de los transistores bipolares: \par

\begin{itemize}
	\item Trabaja con tensión. \par

	\item Tiempos de conmutación bajos. \par

	\item Disipación mucho mayor (como los bipolares). 
\end{itemize}\par

Nos interesa que el transistor se parezca, lo más posible, a un elemento ideal: \par

\begin{itemize}
	\item Pequeñas fugas. \par

	\item Alta potencia. \par

	\item Bajos tiempos de respuesta (ton , toff), para conseguir una alta frecuencia de funcionamiento. \par

	\item Alta concentración de intensidad por unidad de superficie del semiconductor. \par

	\item Que el efecto avalancha se produzca a un valor elevado ( VCE máxima elevada). \par

	\item Que no se produzcan puntos calientes (grandes di/dt ). 
\end{itemize}\par

Una limitación importante de todos los dispositivos de potencia y concretamente de los transistores bipolares, es que el paso de bloqueo a conducción y viceversa no se hace instantáneamente, sino que siempre hay un retardo (ton , toff). Las causas fundamentales de estos retardos son las capacidades asociadas a las uniones colector - base y base - emisor y los tiempos de difusión y recombinación de los portadores. \par


\vspace{\baselineskip}
\textbf{PRINCIPIOS BÁSICOS DE FUNCIONAMIENTO}\par

La diferencia entre un transistor bipolar y un transistor unipolar o FET es el modo de actuación sobre el terminal de control. En el transistor bipolar hay que inyectar una corriente de base para regular la corriente de colector, mientras que en el FET el control se hace mediante la aplicación de una tensión entre puerta y fuente. Esta diferencia vienen determinada por la estructura interna de ambos dispositivos, que son substancialmente distintas. \par

Es una característica común, sin embargo, el hecho de que la potencia que consume el terminal de control (base o puerta) es siempre más pequeña que la potencia manejada en los otros dos terminales. \par

En resumen, destacamos tres cosas fundamentales: \par

\begin{itemize}
	\item En un transistor bipolar I{\fontsize{9pt}{10.8pt}\selectfont B controla la magnitud de IC. \par}\par

	\item En un FET, la tensión V{\fontsize{9pt}{10.8pt}\selectfont GS controla la corriente ID. \par}\par

	\item En ambos casos, con una potencia pequeña puede controlarse otra bastante mayor.\par


\vspace{\baselineskip}

\vspace{\baselineskip}

\vspace{\baselineskip}
\begin{enumerate}
	\item {\fontsize{14pt}{16.8pt}\selectfont \textbf{BIBLIOGRAFÍA:}\par}\par

	\item \href{http://ingenio-triana.blogspot.com/2016/06/eleccion-de-un-transistor-calcular-la.html}{{\fontsize{14pt}{16.8pt}\selectfont ingenio-triana.blogspot.com › 2016/06 › eleccion-de-un-transistor-calcular-la}\par}\par

	\item \href{https://www.uv.es/marinjl/electro/transistores.html}{{\fontsize{14pt}{16.8pt}\selectfont https://www.uv.es › marinjl › electro › transistores}\par}
\end{enumerate}\par

\setlength{\parskip}{6.96pt}
	\item  \par


\end{itemize}
\printbibliography

\end{document}